% <--- percent sign starts a comment line in LaTeX

%----------------------------------------------------------
% This is a sample assignment .tex file. Put your name,
% assignment number and the due date below, as shown.
% Before you typeset your own assignment try to preview 
% and print this one as follows:
%	0. Make sure your LaTeX is installed
%   1. Save this in a file, say HwSample.tex
%	2. Save macros.tex (and other source files, if needed)
%	3. Check the references to other source files in this file, make sure
%		that they are in correct directories. Typically, I will have figures
%		in the same directory, while macros.tex two levels up. But you can
%		change it as you wish.
%   4. Run LaTeX on HwSample.tex
%	5. LaTeX will produce several files, including one file HwSample.pdf
%	6. Use your pdf viewer to view/print HwSample.pdf
% ------------------------------------------------------------

\documentclass[11pt]{article}

\usepackage{fullpage,graphicx,latexsym,picinpar,amsbsy,amsmath,amsfonts}
\usepackage{scrextend}
\usepackage{lipsum}% for demo only!
\usepackage{listings}
\usepackage{verbatimbox}

\input{macros.tex}

\begin{document}
	
% v -- YOUR NAME and SID in the braces
\student{Juan Ruelas}{861116223}  
% v -- ASSIGNMENT NUMBER in the braces
\assignment{1} 
% v-- DUE DATE in the braces
\duedate{Friday, June 3}  

\medskip

%%%%%%%%%%%%%%%%%%%%%%%%%%%%%%%%%%%%%%%%%%%%%%%%%%%%%%%%%%%%%%%%%%%%%%%%%%

\lineacross
%%%%%%%%%%%%%%%%%%%%%%%%%%%%


\begin{solution} \textbf{Consulting Firm} \\

	\hfil
	
	\textbf{A}\\
	
	\hfil
	
	If our moving cost $M = 10$ and the number of operational months $n = 4$, then we have the table below to analyze.
		
		
	\begin{center}
		\begin{tabular}{ | c | c | c | c | c |}
			 \hline  
			 & Month1 & Month2 & Month3 & Month4 \\
			\hline
 			NY & 1 & 3 	& 20 & 30\\ 
 			\hline
 			SF & 50 & 20 & 2 & 4\\  
 			\hline  
		\end{tabular}
	\end{center}
	
	We are given the optimal plan already, however
	 
	\textbf{B}\\
	
	\textbf{C}\\
	
	\textbf{D}\\
\end{solution}


\newpage
\begin{solution} \textbf{Pretty Print}\\

	The entire basis of this problem is to be able to take some text that is "not balanced" and turn it into text whose right margin is as ”even” as possible. Look below to see what I mean.
		
		\hfil
		
		\begin{center}
			\begin{tabular}{ c c c}
 				\includegraphics[width=7cm]{images/unbalanced} & \rightarrow &  \includegraphics[width=7cm]{images/balanced}
			\end{tabular}
		\end{center}
		
	\hfil
	
	In order to accomplish this we will need to make use of dynamic programming. Here is the overview of how we will make use of this programming technique:
	
	\hfil
	
	\begin{itemize}
  		\item Find a recurrence relation for the optimal solution
  		\item Based on our recurrence relation, create an algorithm to solve our problem
	\end{itemize}
	
	
	\hfil


	\textbf{A: Recurrence Relation} \\
	
	In order to come up with a recurrence relation we need to understand what it is we are exactly computing. We are trying to re-arrange the text such that the "slack" or amount of spaces from the last word of every line are evenly distributed among the entirety of the text. This becomes a trivial task until we define what "even" means. Let us assume that "even" means to minimize the sum of all the "slacks". If we were to take this approach we would be left with several different viable solutions. See below for more details. Assume we have a max row width of 10.
	
	\hfil
	
		\begin{center}
			\begin{verbatim}
			                SOLUTION 1                             SOLUTION 2
			
				          Line  0123456789      Slack            Line  0123456789     Slack
				          ---------------------------            --------------------------
				           0    Ruelas      -->  4                0    Ruelas    -->   4
				           1    Juan is my  -->  0       VS       1    Juan is   -->   3
				           2    name        -->  6                2    my name   -->   3
				           		                    
				              4 + 0 + 6 = 10                         4 + 3 + 3 = 10                                     	
			\end{verbatim}				
		\end{center} 
		
		As you can see from the figure above, since we define "even" to be the minimization of all the slacks of every line, we will have multiple "optimal" solutions. This is a problem since we can reproduce identical slack summations with different text patterns and as we can visually see, the pattern on the right appears to be more "even" than the left one. Because of this, we will define "even" to mean the summation of all the $slacks^2$. This will enable us to be greedy with our spaces and will force us to minimize the amount of slack on $every$ line. See below to see what I mean.
		
		\hfil
		
		
		\begin{center}
			\begin{tabular}{c c c}
			    \textbf{SOLUTION 1}: 4 + 0 + 6 = 10 & VS & $4^2$ + $0^2$ + $6^2$ = 52 \\
				\textbf{SOLUTION 2}: 4 + 3 + 3 = 10 & VS & $4^2$ + $3^2$ + $3^2$ = 34
			\end{tabular}
		\end{center}
		
		Therefore we can see that the optimal solution which minimizes the amount of slack on every line is the second solution. We will use this property to determine our recurrence relation.  
		
		
		
		\hfil 
		
		\textbf{Slack Squared}\\
		
		Since we can see that by calculating the summation of the slacks of each line will give us an  "optimal" solution, 
	
	\hfil
	
	\textbf{B: The Algorithm} \\
	In order to solve this problem we will divide this into two sub problems. The 1st subproblem will deal with using dynamic programming to create a table of slack lengths knows as our $slack-cost-table$. The second problem will deal with using the slack cost table to actually calculate the 
	
		
	
\end{solution}



\begin{solution}
	
\end{solution}

\end{document}









